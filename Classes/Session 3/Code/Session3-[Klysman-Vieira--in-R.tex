% Options for packages loaded elsewhere
\PassOptionsToPackage{unicode}{hyperref}
\PassOptionsToPackage{hyphens}{url}
%
\documentclass[
]{article}
\usepackage{amsmath,amssymb}
\usepackage{iftex}
\ifPDFTeX
  \usepackage[T1]{fontenc}
  \usepackage[utf8]{inputenc}
  \usepackage{textcomp} % provide euro and other symbols
\else % if luatex or xetex
  \usepackage{unicode-math} % this also loads fontspec
  \defaultfontfeatures{Scale=MatchLowercase}
  \defaultfontfeatures[\rmfamily]{Ligatures=TeX,Scale=1}
\fi
\usepackage{lmodern}
\ifPDFTeX\else
  % xetex/luatex font selection
\fi
% Use upquote if available, for straight quotes in verbatim environments
\IfFileExists{upquote.sty}{\usepackage{upquote}}{}
\IfFileExists{microtype.sty}{% use microtype if available
  \usepackage[]{microtype}
  \UseMicrotypeSet[protrusion]{basicmath} % disable protrusion for tt fonts
}{}
\makeatletter
\@ifundefined{KOMAClassName}{% if non-KOMA class
  \IfFileExists{parskip.sty}{%
    \usepackage{parskip}
  }{% else
    \setlength{\parindent}{0pt}
    \setlength{\parskip}{6pt plus 2pt minus 1pt}}
}{% if KOMA class
  \KOMAoptions{parskip=half}}
\makeatother
\usepackage{xcolor}
\usepackage[margin=1in]{geometry}
\usepackage{color}
\usepackage{fancyvrb}
\newcommand{\VerbBar}{|}
\newcommand{\VERB}{\Verb[commandchars=\\\{\}]}
\DefineVerbatimEnvironment{Highlighting}{Verbatim}{commandchars=\\\{\}}
% Add ',fontsize=\small' for more characters per line
\usepackage{framed}
\definecolor{shadecolor}{RGB}{248,248,248}
\newenvironment{Shaded}{\begin{snugshade}}{\end{snugshade}}
\newcommand{\AlertTok}[1]{\textcolor[rgb]{0.94,0.16,0.16}{#1}}
\newcommand{\AnnotationTok}[1]{\textcolor[rgb]{0.56,0.35,0.01}{\textbf{\textit{#1}}}}
\newcommand{\AttributeTok}[1]{\textcolor[rgb]{0.13,0.29,0.53}{#1}}
\newcommand{\BaseNTok}[1]{\textcolor[rgb]{0.00,0.00,0.81}{#1}}
\newcommand{\BuiltInTok}[1]{#1}
\newcommand{\CharTok}[1]{\textcolor[rgb]{0.31,0.60,0.02}{#1}}
\newcommand{\CommentTok}[1]{\textcolor[rgb]{0.56,0.35,0.01}{\textit{#1}}}
\newcommand{\CommentVarTok}[1]{\textcolor[rgb]{0.56,0.35,0.01}{\textbf{\textit{#1}}}}
\newcommand{\ConstantTok}[1]{\textcolor[rgb]{0.56,0.35,0.01}{#1}}
\newcommand{\ControlFlowTok}[1]{\textcolor[rgb]{0.13,0.29,0.53}{\textbf{#1}}}
\newcommand{\DataTypeTok}[1]{\textcolor[rgb]{0.13,0.29,0.53}{#1}}
\newcommand{\DecValTok}[1]{\textcolor[rgb]{0.00,0.00,0.81}{#1}}
\newcommand{\DocumentationTok}[1]{\textcolor[rgb]{0.56,0.35,0.01}{\textbf{\textit{#1}}}}
\newcommand{\ErrorTok}[1]{\textcolor[rgb]{0.64,0.00,0.00}{\textbf{#1}}}
\newcommand{\ExtensionTok}[1]{#1}
\newcommand{\FloatTok}[1]{\textcolor[rgb]{0.00,0.00,0.81}{#1}}
\newcommand{\FunctionTok}[1]{\textcolor[rgb]{0.13,0.29,0.53}{\textbf{#1}}}
\newcommand{\ImportTok}[1]{#1}
\newcommand{\InformationTok}[1]{\textcolor[rgb]{0.56,0.35,0.01}{\textbf{\textit{#1}}}}
\newcommand{\KeywordTok}[1]{\textcolor[rgb]{0.13,0.29,0.53}{\textbf{#1}}}
\newcommand{\NormalTok}[1]{#1}
\newcommand{\OperatorTok}[1]{\textcolor[rgb]{0.81,0.36,0.00}{\textbf{#1}}}
\newcommand{\OtherTok}[1]{\textcolor[rgb]{0.56,0.35,0.01}{#1}}
\newcommand{\PreprocessorTok}[1]{\textcolor[rgb]{0.56,0.35,0.01}{\textit{#1}}}
\newcommand{\RegionMarkerTok}[1]{#1}
\newcommand{\SpecialCharTok}[1]{\textcolor[rgb]{0.81,0.36,0.00}{\textbf{#1}}}
\newcommand{\SpecialStringTok}[1]{\textcolor[rgb]{0.31,0.60,0.02}{#1}}
\newcommand{\StringTok}[1]{\textcolor[rgb]{0.31,0.60,0.02}{#1}}
\newcommand{\VariableTok}[1]{\textcolor[rgb]{0.00,0.00,0.00}{#1}}
\newcommand{\VerbatimStringTok}[1]{\textcolor[rgb]{0.31,0.60,0.02}{#1}}
\newcommand{\WarningTok}[1]{\textcolor[rgb]{0.56,0.35,0.01}{\textbf{\textit{#1}}}}
\usepackage{graphicx}
\makeatletter
\def\maxwidth{\ifdim\Gin@nat@width>\linewidth\linewidth\else\Gin@nat@width\fi}
\def\maxheight{\ifdim\Gin@nat@height>\textheight\textheight\else\Gin@nat@height\fi}
\makeatother
% Scale images if necessary, so that they will not overflow the page
% margins by default, and it is still possible to overwrite the defaults
% using explicit options in \includegraphics[width, height, ...]{}
\setkeys{Gin}{width=\maxwidth,height=\maxheight,keepaspectratio}
% Set default figure placement to htbp
\makeatletter
\def\fps@figure{htbp}
\makeatother
\setlength{\emergencystretch}{3em} % prevent overfull lines
\providecommand{\tightlist}{%
  \setlength{\itemsep}{0pt}\setlength{\parskip}{0pt}}
\setcounter{secnumdepth}{-\maxdimen} % remove section numbering
\ifLuaTeX
  \usepackage{selnolig}  % disable illegal ligatures
\fi
\IfFileExists{bookmark.sty}{\usepackage{bookmark}}{\usepackage{hyperref}}
\IfFileExists{xurl.sty}{\usepackage{xurl}}{} % add URL line breaks if available
\urlstyle{same}
\hypersetup{
  pdftitle={R Notebook},
  hidelinks,
  pdfcreator={LaTeX via pandoc}}

\title{R Notebook}
\author{}
\date{\vspace{-2.5em}}

\begin{document}
\maketitle

\hypertarget{eda-singapore-accommodations}{%
\section{EDA Singapore
Accommodations}\label{eda-singapore-accommodations}}

\hypertarget{introduction}{%
\section{Introduction:}\label{introduction}}

During exploratory data analysis (EDA), the dataframe that has
information about accommodation in Singapore was worked on. Some
dataframe columns contain the following information:

\begin{itemize}
\tightlist
\item
  \texttt{neighborhood\_group}: Indicates the region of the neighborhood
  where the accommodation is located.
\item
  \texttt{name}: Name of the accommodation.
\item
  \texttt{host\_id}: ID of the host responsible for the accommodation.
\item
  \texttt{host\_name}: Host name.
\item
  \texttt{neighborhood}: Name of the neighborhood where the
  accommodation is located.
\item
  \texttt{latitude} and \texttt{longitude}: Geographic coordinates of
  the accommodation.
\item
  \texttt{number\_of\_reviews}: Number of reviews received by the
  accommodation.
\item
  \texttt{room\_type}: Type of room available in the accommodation.
\item
  \texttt{price}: Price of the accommodation.
\item
  \texttt{availability\_365}: Number of days the accommodation is
  available to book over the course of a year.
\end{itemize}

During exploratory data analysis, we explore and summarize the main
characteristics and patterns present in these data.

\begin{enumerate}
\def\labelenumi{\arabic{enumi}.}
\item
  Descriptive analysis: We perform a statistical description of the
  variables, such as mean, median, standard deviation, minimum and
  maximum. This helps us understand the distribution of data and
  identify potential discrepancies or outliers.
\item
  Handling missing or inconsistent data: We identify and handle missing
  or inconsistent data, such as nulls or outliers. This may involve
  deleting records with missing data or filling in these values with
  appropriate techniques.
\item
  Data visualization: We use graphs and visualizations to represent data
  in a more understandable way. For example, we can create bar charts to
  show the distribution of room types or a map to visualize the location
  of accommodations.
\end{enumerate}

\hypertarget{insights}{%
\section{Insights}\label{insights}}

\begin{itemize}
\item
  We look for interesting patterns and insights in the data that can be
  useful for decision-making or answering specific questions. For
  example, we may discover that certain neighborhoods have a greater
  availability of accommodation throughout the year or that certain room
  types have a direct relationship with price.
\item
  There are relationships between the variables present in the
  dataframe. For example, we can analyze whether the price of
  accommodation varies according to the region of the neighborhood or
  whether there is any correlation between the number of reviews and
  availability throughout the year.
\end{itemize}

\hypertarget{loading-the-raw-dataframe}{%
\subsection{Loading the raw dataframe}\label{loading-the-raw-dataframe}}

\begin{Shaded}
\begin{Highlighting}[]
\FunctionTok{library}\NormalTok{(readxl)}

\NormalTok{file\_path }\OtherTok{\textless{}{-}} \StringTok{"G:/My Drive/IPS/Mestrado/UCs/Aprendizage supervisionada/IPS{-}ESCE SP ML/Session 3/Data/Singapore Airbnb {-} Raw.xlsx"}

\NormalTok{data }\OtherTok{\textless{}{-}} \FunctionTok{read\_excel}\NormalTok{(file\_path)}

\NormalTok{num\_rows }\OtherTok{\textless{}{-}} \FunctionTok{dim}\NormalTok{(data)[}\DecValTok{1}\NormalTok{]}
\NormalTok{num\_cols }\OtherTok{\textless{}{-}} \FunctionTok{dim}\NormalTok{(data)[}\DecValTok{2}\NormalTok{]}

\NormalTok{column\_names }\OtherTok{\textless{}{-}} \FunctionTok{colnames}\NormalTok{(data)}

\FunctionTok{cat}\NormalTok{(}\StringTok{"Number of rows:"}\NormalTok{, num\_rows, }\StringTok{"}\SpecialCharTok{\textbackslash{}n}\StringTok{"}\NormalTok{)}
\end{Highlighting}
\end{Shaded}

\begin{verbatim}
## Number of rows: 7923
\end{verbatim}

\begin{Shaded}
\begin{Highlighting}[]
\FunctionTok{cat}\NormalTok{(}\StringTok{"Number of columns:"}\NormalTok{, num\_cols, }\StringTok{"}\SpecialCharTok{\textbackslash{}n}\StringTok{"}\NormalTok{)}
\end{Highlighting}
\end{Shaded}

\begin{verbatim}
## Number of columns: 16
\end{verbatim}

\begin{Shaded}
\begin{Highlighting}[]
\FunctionTok{cat}\NormalTok{(}\StringTok{"Column names:"}\NormalTok{, }\FunctionTok{paste}\NormalTok{(column\_names, }\AttributeTok{collapse =} \StringTok{", "}\NormalTok{), }\StringTok{"}\SpecialCharTok{\textbackslash{}n}\StringTok{"}\NormalTok{)}
\end{Highlighting}
\end{Shaded}

\begin{verbatim}
## Column names: id, name, host_id, host_name, neighbourhood_group, neighbourhood, latitude, longitude, room_type, price, minimum_nights, number_of_reviews, last_review, reviews_per_month, calculated_host_listings_count, availability_365
\end{verbatim}

\hypertarget{checking-for-missing-data}{%
\subsection{Checking for missing data}\label{checking-for-missing-data}}

\begin{Shaded}
\begin{Highlighting}[]
\NormalTok{missing\_values }\OtherTok{\textless{}{-}} \FunctionTok{colSums}\NormalTok{(}\FunctionTok{is.na}\NormalTok{(data))}

\FunctionTok{print}\NormalTok{(}\StringTok{"Number of missing values in each column:"}\NormalTok{)}
\end{Highlighting}
\end{Shaded}

\begin{verbatim}
## [1] "Number of missing values in each column:"
\end{verbatim}

\begin{Shaded}
\begin{Highlighting}[]
\FunctionTok{print}\NormalTok{(missing\_values)}
\end{Highlighting}
\end{Shaded}

\begin{verbatim}
##                             id                           name 
##                              0                              5 
##                        host_id                      host_name 
##                             13                              0 
##            neighbourhood_group                  neighbourhood 
##                              0                              0 
##                       latitude                      longitude 
##                             20                              9 
##                      room_type                          price 
##                              0                              0 
##                 minimum_nights              number_of_reviews 
##                              0                              4 
##                    last_review              reviews_per_month 
##                           2773                           2773 
## calculated_host_listings_count               availability_365 
##                              0                              0
\end{verbatim}

\hypertarget{display-the-first-rows-of-data}{%
\subsection{Display the first rows of
data}\label{display-the-first-rows-of-data}}

\begin{Shaded}
\begin{Highlighting}[]
\FunctionTok{head}\NormalTok{(data)}
\end{Highlighting}
\end{Shaded}

\begin{verbatim}
## # A tibble: 6 x 16
##      id name        host_id host_name neighbourhood_group neighbourhood latitude
##   <dbl> <chr>         <dbl> <chr>     <chr>               <chr>            <dbl>
## 1 49091 COZICOMFOR~  266763 Francesca North Region        Woodlands         1.44
## 2 50646 Pleasant R~  227796 Sujatha   Central Region      Bukit Timah       1.33
## 3 56334 COZICOMFORT  266763 Francesca North Region        Woodlands         1.44
## 4 71609 Ensuite Ro~  367042 Belinda   East Region         Tampines          1.35
## 5 71896 B&B  Room ~  367042 Belinda   East Region         Tampines          1.35
## 6 71903 Room 2-nea~  367042 Belinda   East Region         Tampines          1.35
## # i 9 more variables: longitude <dbl>, room_type <chr>, price <dbl>,
## #   minimum_nights <dbl>, number_of_reviews <dbl>, last_review <dttm>,
## #   reviews_per_month <dbl>, calculated_host_listings_count <dbl>,
## #   availability_365 <dbl>
\end{verbatim}

\hypertarget{statistical-data-about-the-dataframe}{%
\subsection{Statistical data about the
dataframe}\label{statistical-data-about-the-dataframe}}

\begin{Shaded}
\begin{Highlighting}[]
\FunctionTok{summary}\NormalTok{(data)}
\end{Highlighting}
\end{Shaded}

\begin{verbatim}
##        id               name              host_id           host_name        
##  Min.   :   49091   Length:7923        Min.   :    23666   Length:7923       
##  1st Qu.:15824582   Class :character   1st Qu.: 23129381   Class :character  
##  Median :24707713   Mode  :character   Median : 63448912   Mode  :character  
##  Mean   :23404133                      Mean   : 91313156                     
##  3rd Qu.:32365800                      3rd Qu.:155656938                     
##  Max.   :38112762                      Max.   :288567551                     
##                                        NA's   :13                            
##  neighbourhood_group neighbourhood         latitude       longitude    
##  Length:7923         Length:7923        Min.   :1.244   Min.   :103.6  
##  Class :character    Class :character   1st Qu.:1.296   1st Qu.:103.8  
##  Mode  :character    Mode  :character   Median :1.311   Median :103.8  
##                                         Mean   :1.314   Mean   :103.8  
##                                         3rd Qu.:1.322   3rd Qu.:103.9  
##                                         Max.   :1.455   Max.   :104.0  
##                                         NA's   :20      NA's   :9      
##   room_type             price         minimum_nights    number_of_reviews
##  Length:7923        Min.   :    0.0   Min.   :   1.00   Min.   :  0.00   
##  Class :character   1st Qu.:   65.0   1st Qu.:   1.00   1st Qu.:  0.00   
##  Mode  :character   Median :  124.0   Median :   3.00   Median :  2.00   
##                     Mean   :  192.3   Mean   :  17.53   Mean   : 12.73   
##                     3rd Qu.:  199.0   3rd Qu.:  10.00   3rd Qu.: 10.00   
##                     Max.   :65000.0   Max.   :1000.00   Max.   :323.00   
##                                                         NA's   :4        
##   last_review                     reviews_per_month
##  Min.   :2013-10-21 00:00:00.00   Min.   : 0.010   
##  1st Qu.:2018-11-21 00:00:00.00   1st Qu.: 0.180   
##  Median :2019-06-27 00:00:00.00   Median : 0.550   
##  Mean   :2019-01-11 17:33:01.04   Mean   : 1.044   
##  3rd Qu.:2019-08-07 00:00:00.00   3rd Qu.: 1.370   
##  Max.   :2019-08-27 00:00:00.00   Max.   :13.000   
##  NA's   :2773                     NA's   :2773     
##  calculated_host_listings_count availability_365
##  Min.   :  1.00                 Min.   :  0.0   
##  1st Qu.:  2.00                 1st Qu.: 54.0   
##  Median :  9.00                 Median :260.0   
##  Mean   : 40.55                 Mean   :208.7   
##  3rd Qu.: 48.00                 3rd Qu.:355.0   
##  Max.   :274.00                 Max.   :365.0   
## 
\end{verbatim}

\hypertarget{data-cleaning}{%
\section{Data cleaning}\label{data-cleaning}}

\begin{Shaded}
\begin{Highlighting}[]
\CommentTok{\#Remover colunas desnecessárias}
\NormalTok{data }\OtherTok{\textless{}{-}}\NormalTok{ data[, }\FunctionTok{c}\NormalTok{(}\StringTok{"neighbourhood\_group"}\NormalTok{, }\StringTok{"name"}\NormalTok{, }\StringTok{"host\_id"}\NormalTok{, }\StringTok{"host\_name"}\NormalTok{, }\StringTok{"neighbourhood"}\NormalTok{, }\StringTok{"latitude"}\NormalTok{, }\StringTok{"number\_of\_reviews"}\NormalTok{, }\StringTok{"longitude"}\NormalTok{, }\StringTok{"room\_type"}\NormalTok{, }\StringTok{"price"}\NormalTok{, }\StringTok{"availability\_365"}\NormalTok{)]}
\end{Highlighting}
\end{Shaded}

\hypertarget{filter-only-complete-rows-no-null-values}{%
\subsection{Filter only complete rows (no null
values)}\label{filter-only-complete-rows-no-null-values}}

\begin{Shaded}
\begin{Highlighting}[]
\NormalTok{data\_clean }\OtherTok{\textless{}{-}}\NormalTok{ data[}\FunctionTok{complete.cases}\NormalTok{(data), ]}
\end{Highlighting}
\end{Shaded}

\hypertarget{identify-and-remove-duplicate-elements}{%
\subsection{Identify and remove duplicate
elements}\label{identify-and-remove-duplicate-elements}}

\begin{Shaded}
\begin{Highlighting}[]
\NormalTok{data\_unique }\OtherTok{\textless{}{-}} \FunctionTok{subset}\NormalTok{(data\_clean, }\SpecialCharTok{!}\FunctionTok{duplicated}\NormalTok{(data\_clean))}
\end{Highlighting}
\end{Shaded}

\hypertarget{boxplot-to-understand-the-distribution-of-data-depending-on-the-price-column}{%
\subsection{Boxplot to understand the distribution of data depending on
the price
column}\label{boxplot-to-understand-the-distribution-of-data-depending-on-the-price-column}}

\begin{Shaded}
\begin{Highlighting}[]
\CommentTok{\# Seleciona as colunas desejadas para criar os boxplots}
\FunctionTok{boxplot}\NormalTok{(data\_unique}\SpecialCharTok{$}\NormalTok{price, }\AttributeTok{col =} \StringTok{"green"}\NormalTok{, }\AttributeTok{ylim =} \FunctionTok{c}\NormalTok{(}\DecValTok{0}\NormalTok{, }\DecValTok{1000}\NormalTok{))}
\end{Highlighting}
\end{Shaded}

\includegraphics{Session3-{[}Klysman-Vieira--in-R_files/figure-latex/unnamed-chunk-8-1.pdf}

\hypertarget{remove-lines-outliers-where-price-500}{%
\subsection{Remove lines (outliers) where price \textgreater=
500}\label{remove-lines-outliers-where-price-500}}

\begin{Shaded}
\begin{Highlighting}[]
\NormalTok{data\_filtered\_500 }\OtherTok{\textless{}{-}} \FunctionTok{subset}\NormalTok{(data\_unique, price }\SpecialCharTok{\textless{}} \DecValTok{500}\NormalTok{)}
\end{Highlighting}
\end{Shaded}

\hypertarget{graphics}{%
\section{Graphics}\label{graphics}}

\hypertarget{create-a-boxplot-for-the-price-column}{%
\subsection{Create a boxplot for the price
column}\label{create-a-boxplot-for-the-price-column}}

\begin{Shaded}
\begin{Highlighting}[]
\NormalTok{boxplot\_output }\OtherTok{\textless{}{-}} \FunctionTok{boxplot}\NormalTok{(data\_filtered\_500}\SpecialCharTok{$}\NormalTok{price, }\AttributeTok{col =} \StringTok{"green"}\NormalTok{, }\AttributeTok{ylim =} \FunctionTok{c}\NormalTok{(}\DecValTok{0}\NormalTok{, }\DecValTok{500}\NormalTok{))}
\end{Highlighting}
\end{Shaded}

\includegraphics{Session3-{[}Klysman-Vieira--in-R_files/figure-latex/unnamed-chunk-10-1.pdf}

\begin{Shaded}
\begin{Highlighting}[]
\CommentTok{\# Acessa os resultados matemáticos do boxplot}
\NormalTok{min\_value }\OtherTok{\textless{}{-}}\NormalTok{ boxplot\_output}\SpecialCharTok{$}\NormalTok{stats[}\DecValTok{1}\NormalTok{]}
\NormalTok{q1 }\OtherTok{\textless{}{-}}\NormalTok{ boxplot\_output}\SpecialCharTok{$}\NormalTok{stats[}\DecValTok{2}\NormalTok{]}
\NormalTok{median }\OtherTok{\textless{}{-}}\NormalTok{ boxplot\_output}\SpecialCharTok{$}\NormalTok{stats[}\DecValTok{3}\NormalTok{]}
\NormalTok{q3 }\OtherTok{\textless{}{-}}\NormalTok{ boxplot\_output}\SpecialCharTok{$}\NormalTok{stats[}\DecValTok{4}\NormalTok{]}
\NormalTok{max\_value }\OtherTok{\textless{}{-}}\NormalTok{ boxplot\_output}\SpecialCharTok{$}\NormalTok{stats[}\DecValTok{5}\NormalTok{]}

\CommentTok{\# Imprime os resultados}
\FunctionTok{cat}\NormalTok{(}\StringTok{"Valor mínimo:"}\NormalTok{, min\_value, }\StringTok{"}\SpecialCharTok{\textbackslash{}n}\StringTok{"}\NormalTok{)}
\end{Highlighting}
\end{Shaded}

\begin{verbatim}
## Valor mínimo: 0
\end{verbatim}

\begin{Shaded}
\begin{Highlighting}[]
\FunctionTok{cat}\NormalTok{(}\StringTok{"Primeiro quartil:"}\NormalTok{, q1, }\StringTok{"}\SpecialCharTok{\textbackslash{}n}\StringTok{"}\NormalTok{)}
\end{Highlighting}
\end{Shaded}

\begin{verbatim}
## Primeiro quartil: 65
\end{verbatim}

\begin{Shaded}
\begin{Highlighting}[]
\FunctionTok{cat}\NormalTok{(}\StringTok{"Mediana:"}\NormalTok{, median, }\StringTok{"}\SpecialCharTok{\textbackslash{}n}\StringTok{"}\NormalTok{)}
\end{Highlighting}
\end{Shaded}

\begin{verbatim}
## Mediana: 119
\end{verbatim}

\begin{Shaded}
\begin{Highlighting}[]
\FunctionTok{cat}\NormalTok{(}\StringTok{"Terceiro quartil:"}\NormalTok{, q3, }\StringTok{"}\SpecialCharTok{\textbackslash{}n}\StringTok{"}\NormalTok{)}
\end{Highlighting}
\end{Shaded}

\begin{verbatim}
## Terceiro quartil: 187
\end{verbatim}

\begin{Shaded}
\begin{Highlighting}[]
\FunctionTok{cat}\NormalTok{(}\StringTok{"Valor máximo:"}\NormalTok{, max\_value, }\StringTok{"}\SpecialCharTok{\textbackslash{}n}\StringTok{"}\NormalTok{)}
\end{Highlighting}
\end{Shaded}

\begin{verbatim}
## Valor máximo: 369
\end{verbatim}

\hypertarget{create-a-bar-chart-for-the-room_type-column}{%
\subsection{Create a bar chart for the ``room\_type''
column}\label{create-a-bar-chart-for-the-room_type-column}}

\begin{Shaded}
\begin{Highlighting}[]
\FunctionTok{barplot}\NormalTok{(}\FunctionTok{table}\NormalTok{(data\_filtered\_500}\SpecialCharTok{$}\NormalTok{room\_type), }\AttributeTok{main =} \StringTok{"Gráfico de Barras {-} Room Type"}\NormalTok{, }\AttributeTok{xlab =} \StringTok{"Room Type"}\NormalTok{, }\AttributeTok{ylab =} \StringTok{"Frequência"}\NormalTok{)}
\end{Highlighting}
\end{Shaded}

\includegraphics{Session3-{[}Klysman-Vieira--in-R_files/figure-latex/unnamed-chunk-11-1.pdf}

\hypertarget{creates-a-bar-chart-for-the-room_type-column.-shows-the-regions-of-greatest-interest-in-singapore}{%
\subsection{Creates a bar chart for the ``room\_type'' column. Shows the
regions of greatest interest in
Singapore}\label{creates-a-bar-chart-for-the-room_type-column.-shows-the-regions-of-greatest-interest-in-singapore}}

\begin{Shaded}
\begin{Highlighting}[]
\FunctionTok{barplot}\NormalTok{(}\FunctionTok{table}\NormalTok{(data\_filtered\_500}\SpecialCharTok{$}\NormalTok{neighbourhood\_group), }\AttributeTok{main =} \StringTok{"Gráfico de Barras {-} neighbourhood\_group"}\NormalTok{, }\AttributeTok{xlab =} \StringTok{"neighbourhood\_group"}\NormalTok{, }\AttributeTok{ylab =} \StringTok{"Frequência"}\NormalTok{)}
\end{Highlighting}
\end{Shaded}

\includegraphics{Session3-{[}Klysman-Vieira--in-R_files/figure-latex/unnamed-chunk-12-1.pdf}

\begin{Shaded}
\begin{Highlighting}[]
\CommentTok{\# Obtém o índice da linha com o maior valor em number\_of\_reviews}
\NormalTok{index }\OtherTok{\textless{}{-}} \FunctionTok{which.max}\NormalTok{(data\_filtered\_500}\SpecialCharTok{$}\NormalTok{number\_of\_reviews)}

\CommentTok{\# Obtém a linha completa com o maior valor em number\_of\_reviews}
\NormalTok{row\_with\_max\_reviews }\OtherTok{\textless{}{-}}\NormalTok{ data\_filtered\_500[index, ]}

\CommentTok{\# Imprime a linha completa}
\FunctionTok{print}\NormalTok{(row\_with\_max\_reviews)}
\end{Highlighting}
\end{Shaded}

\begin{verbatim}
## # A tibble: 1 x 11
##   neighbourhood_group name              host_id host_name neighbourhood latitude
##   <chr>               <chr>               <dbl> <chr>     <chr>            <dbl>
## 1 East Region         Luxuriously Spac~ 7642747 Shirley   Bedok             1.32
## # i 5 more variables: number_of_reviews <dbl>, longitude <dbl>,
## #   room_type <chr>, price <dbl>, availability_365 <dbl>
\end{verbatim}

\hypertarget{ggplot2}{%
\section{GGPLOT2}\label{ggplot2}}

\hypertarget{bar-graphs}{%
\subsubsection{Bar graphs}\label{bar-graphs}}

\begin{Shaded}
\begin{Highlighting}[]
\FunctionTok{library}\NormalTok{(ggplot2)}

\CommentTok{\# Cria o gráfico de barras usando ggplot2}
\FunctionTok{ggplot}\NormalTok{(data\_filtered\_500, }\FunctionTok{aes}\NormalTok{(}\AttributeTok{x =}\NormalTok{ room\_type)) }\SpecialCharTok{+}
  \FunctionTok{geom\_bar}\NormalTok{(}\AttributeTok{fill =} \StringTok{"blue"}\NormalTok{, }\AttributeTok{color =} \StringTok{"black"}\NormalTok{) }\SpecialCharTok{+}
  \FunctionTok{labs}\NormalTok{(}\AttributeTok{x =} \StringTok{"Tipo de Quarto"}\NormalTok{, }\AttributeTok{y =} \StringTok{"Contagem"}\NormalTok{) }\SpecialCharTok{+}
  \FunctionTok{ggtitle}\NormalTok{(}\StringTok{"Contagem de Quartos por Tipo"}\NormalTok{)}
\end{Highlighting}
\end{Shaded}

\includegraphics{Session3-{[}Klysman-Vieira--in-R_files/figure-latex/unnamed-chunk-14-1.pdf}

\begin{Shaded}
\begin{Highlighting}[]
\FunctionTok{ggplot}\NormalTok{(data\_filtered\_500, }\FunctionTok{aes}\NormalTok{(}\AttributeTok{x =}\NormalTok{ neighbourhood\_group)) }\SpecialCharTok{+}
  \FunctionTok{geom\_bar}\NormalTok{(}\AttributeTok{fill =} \StringTok{"blue"}\NormalTok{, }\AttributeTok{color =} \StringTok{"black"}\NormalTok{) }\SpecialCharTok{+}
  \FunctionTok{labs}\NormalTok{(}\AttributeTok{x =} \StringTok{"Neighbourhood Group"}\NormalTok{, }\AttributeTok{y =} \StringTok{"Count"}\NormalTok{) }\SpecialCharTok{+}
  \FunctionTok{ggtitle}\NormalTok{(}\StringTok{"Count of Listings by Neighbourhood Group"}\NormalTok{)}
\end{Highlighting}
\end{Shaded}

\includegraphics{Session3-{[}Klysman-Vieira--in-R_files/figure-latex/unnamed-chunk-15-1.pdf}

\begin{Shaded}
\begin{Highlighting}[]
\FunctionTok{ggplot}\NormalTok{(data\_filtered\_500, }\FunctionTok{aes}\NormalTok{(}\AttributeTok{x =}\NormalTok{ price, }\AttributeTok{y =}\NormalTok{ number\_of\_reviews)) }\SpecialCharTok{+}
  \FunctionTok{geom\_point}\NormalTok{() }\SpecialCharTok{+}
  \FunctionTok{labs}\NormalTok{(}\AttributeTok{x =} \StringTok{"Price"}\NormalTok{, }\AttributeTok{y =} \StringTok{"Number of Reviews"}\NormalTok{) }\SpecialCharTok{+}
  \FunctionTok{ggtitle}\NormalTok{(}\StringTok{"Price vs. Number of Reviews"}\NormalTok{)}
\end{Highlighting}
\end{Shaded}

\includegraphics{Session3-{[}Klysman-Vieira--in-R_files/figure-latex/unnamed-chunk-16-1.pdf}

\begin{Shaded}
\begin{Highlighting}[]
\FunctionTok{ggplot}\NormalTok{(data\_filtered\_500, }\FunctionTok{aes}\NormalTok{(}\AttributeTok{x =}\NormalTok{ neighbourhood\_group, }\AttributeTok{y =}\NormalTok{ price)) }\SpecialCharTok{+}
  \FunctionTok{geom\_boxplot}\NormalTok{() }\SpecialCharTok{+}
  \FunctionTok{labs}\NormalTok{(}\AttributeTok{x =} \StringTok{"Neighbourhood Group"}\NormalTok{, }\AttributeTok{y =} \StringTok{"Price"}\NormalTok{) }\SpecialCharTok{+}
  \FunctionTok{ggtitle}\NormalTok{(}\StringTok{"Price Distribution by Neighbourhood Group"}\NormalTok{)}
\end{Highlighting}
\end{Shaded}

\includegraphics{Session3-{[}Klysman-Vieira--in-R_files/figure-latex/unnamed-chunk-17-1.pdf}

\begin{Shaded}
\begin{Highlighting}[]
\FunctionTok{ggplot}\NormalTok{(data\_filtered\_500, }\FunctionTok{aes}\NormalTok{(}\AttributeTok{x =}\NormalTok{ price, }\AttributeTok{fill =}\NormalTok{ room\_type)) }\SpecialCharTok{+}
  \FunctionTok{geom\_density}\NormalTok{(}\AttributeTok{alpha =} \FloatTok{0.5}\NormalTok{) }\SpecialCharTok{+}
  \FunctionTok{labs}\NormalTok{(}\AttributeTok{x =} \StringTok{"Price"}\NormalTok{, }\AttributeTok{y =} \StringTok{"Density"}\NormalTok{) }\SpecialCharTok{+}
  \FunctionTok{ggtitle}\NormalTok{(}\StringTok{"Price Distribution by Room Type"}\NormalTok{)}
\end{Highlighting}
\end{Shaded}

\includegraphics{Session3-{[}Klysman-Vieira--in-R_files/figure-latex/unnamed-chunk-18-1.pdf}

\#salvando o novo dataframe data\_filtered\_500\_price em formato XLSX

\begin{Shaded}
\begin{Highlighting}[]
\FunctionTok{library}\NormalTok{(openxlsx)}
\NormalTok{new\_data }\OtherTok{\textless{}{-}} \StringTok{"G:/My Drive/IPS/Mestrado/UCs/Aprendizage supervisionada/IPS{-}ESCE SP ML/Session 3/Data/data\_filtered\_500\_price.xlsx"}
\FunctionTok{write.xlsx}\NormalTok{(data\_filtered\_500, }\AttributeTok{file =}\NormalTok{ new\_data)}
\end{Highlighting}
\end{Shaded}


\end{document}
